\section{Conclusion}
The case at hand deals with the health effects of wearable devices. While the first study is positive about their impacts, the second study concludes that devices that monitor and provide feedback on physical activity may not offer an advantage over standard behavioral weight loss approaches. John Jakicic, the lead author of this study, says that ``these findings don’t mean that fitness trackers don’t work. They just don’t work for everybody. They’re marketed on the premise that if you see how little you’re doing physically, you’ll be motivated to do more, but that’s a very simplistic way to think about changing behavior. Many people need a lot more than that.'' On the positive side, he continues that ``if you’re very committed to working out, and you’re really into numbers, these devices may be a big help.'' \cite{ref07}

In general, according to John Hopkins Medicine \cite{ref04}, any user of wearable fitness devices should consider five vital points to see their efforts come into fruition: use the tracker daily and consistently; set a goal probably recommended by a doctor; find enjoyable activities that also fit into daily life and can be sustained over the long-term; recruit friends and family to use trackers as well; be accountable by reporting health data and similar information to a responsible individual such as a doctor.
