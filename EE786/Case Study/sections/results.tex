\section{Results}
In general, the first study converges its conclusion upon the effectiveness of health wearable devices in reducing body weight. In particular, the study found commercial health wearable-only and accelerometer/pedometer-only physical activity (PA) interventions to be most effective for reduction in body weight in comparison to multicomponent accelerometer/pedometer interventions. Specifically, interventions which were over 12 weeks in duration were most effective for achieving this outcome.

Interestingly, the second study noted that, among young adults with a body mass index (BMI) between 25 and less than 40, the addition of a wearable technology device to a standard behavioral intervention resulted in a decline in weight loss over 24 months. This decrease in the rate of weight loss makes the second study critical about using fitness trackers. The results of this study however do not lie in contradiction to the first study since the first case concluded in the loss of weight while this study concluded in the decrease of loss of weight.

As noted earlier, the third study deals with the question of wearable fitness device usage. It suggests that consumer-grade smart watches have penetrated the health research space rapidly since 2014. However, smart watch technical function, acceptability, and effectiveness in supporting health must be validated in larger field studies that enroll actual participants living with the conditions these devices target.